\include{settings}

\begin{document} % начало документа
\raggedbottom
%\begin{titlepage}	% начало титульной страницы

	\begin{center}		% выравнивание по центру

		\large \university \\
		\large \faculty \\
		\large \department \\[6cm]
		% название института, затем отступ 6см

		\huge \subject \\[0.5cm] % название работы, затем отступ 0,5см
		\large \docname \num \\[5.1cm]
		% \large Тема работы\\[5cm]

	\end{center}


	\begin{flushright} % выравнивание по правому краю
		\begin{minipage}{0.25\textwidth} % врезка в половину ширины текста
			\begin{flushleft} % выровнять её содержимое по левому краю

				\large\textbf{Работу выполнил:}\\
				\large \studentname \\
				\large {Группа:} \group \\

				\large \textbf{Преподаватель:}\\
				\large \tutorname

			\end{flushleft}
		\end{minipage}
	\end{flushright}

	\vfill % заполнить всё доступное ниже пространство

	\begin{center}
		\large \city \\
		\large \the\year % вывести дату
	\end{center} % закончить выравнивание по центру

\end{titlepage} % конец титульной страницы

\vfill % заполнить всё доступное ниже пространство



% Содержание
%\tableofcontents
%\newpage
\title{Обзор методов SLAM и их ограничений}
\author{Пестов А.}
\institute{Санкт-Петербургский Политехнический Университет Петра Великого
\email{pestov.av@edu.spbstu.ru}}
\maketitle

\begin{abstract}
	% TODO: дополнить  
	В этой статье приводится обзор методов SLAM, их особенности, преимущества и ограничения. 
\end{abstract}

\section{Введение}
Одной из основных задач систем управления автономных роботов является задача навигации и ее подзадача —  локализация робота в пространстве. Одновременная локализация и построение карты — метод, используемый в робототехнике для построения и обновления карты неизвестной среды с одновременным отслеживанием местоположение робота в этой карте. Построение карты необходимо для осуществления других задач, например, по картам робот составляет план маршрута или в зависимости от окружающей обстановки принимает решение, в соответствии с поставленной задачей.

Основными возможностями робота, необходимые для локализации и построение карты, являются показания с сенсоров и возможность сохранять информацию о предыдущих измерениях. С помощью методов визуальной одометрии или на основе анализа дальномерных данных – робот может определять свое смещение относительно предыдущего положения. В идеальном случае – когда его вычисления точны и безукоризнены – по одним этим данным возможно воссоздать карту местности, где он уже побывал и полностью описать траекторию его движения. В реальных условиях на каждом шаге возникает погрешность вычислений, которую необходимо учитывать. Иначе, даже при приемлемой точности определение локального смещения, из-за накопленной ошибки глобальная карта положений робота будет полна искажений. 

В настоящее время существует большой ряд алгоритмов реализации технологии SLAM. Возможности и пределы различных датчиков, а также аппаратные ограничения и повышение эффективности вычислений были основной причиной разработки новых алгоритмов. Популярные алгоритмы используют специальную фильтрацию – расширенный фильтр Калмана, фильтр частиц и статистическое оценивание данных. 

\section{Принцип стереозрения}



\section{Выводы}
\label{conclusion}

Новые алгоритмы для технологии SLAM остаются до сих пор активной областью иследований, которые ведутся согласно различным требования и предложениям о типах карт, датчиков и моделей обработкаю Многие современные системы, реализующие технологию SLAM, оновываются на комбинацию различных подходов.







\newpage
\bibliographystyle{./config/splncs04}
\bibliography{refs}

\end{document}
